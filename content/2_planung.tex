\chapter{Projektplanung}\label{ch:data}
\section{Projektziele}
Die groben Projektziele ergeben sich aus der Projektaufgabe. Konkret umfasst dies die Implementierung der folgenden Entitäten:
\texttt{Tarif}, \texttt{Tarifbaustein}, \texttt{Tarif\-baustein-Pricing}, \texttt{Tafelsystem}, \texttt{Überschuss}, \texttt{Produkt} sowie \texttt{Produkt-Pricing}.

Ein \texttt{Tarif} besteht jeweils aus mehreren \texttt{Tarifbausteinen}. Ein \texttt{Tarifbaustein} wiederum aus mehreren \texttt{Tarifbaustein-Pricings}. \texttt{Tarifbausteine} können null-drei \texttt{Tafelsysteme} und \texttt{Überschüsse} haben.
Ein \texttt{Produkt} besteht aus mehreren \texttt{Produkt-Pricings}. 

Neben den Grundfunktionen wie CRUD (Create, Read, Update, Delete), die alle Entitäten haben sollen, gibt es noch spezielle Funktionen.
Ein Tarif sowohl als auch ein Produkt sollen deaktiviert werden können. Dies dient ebenfalls als Voraussetzung für eine Löschung. Bei der Löschung von Tarifen soll darauf geachtet werden, dass \texttt{Überschüsse} und
Tafelsysteme die keinem Tarifbaustein mehr zugehörig sind auch entfernt werden um keine bezugslose Entitäten zu speichern. 

Dementsprechend sollen alle Kind-Entitäten auch nur mit der Erstellung eines Vater-Elements angelegt werden können.
Bei der Erstellung neuer \texttt{Tarife} soll es ebenfalls möglich sein einen bereits bestehenden \texttt{Tarif} als Vorlage zu verwenden. Hier ist es wichtig, dass nicht nur die direkten Tarifinformationen, 
sondern auch alle Daten der zugehörigen Kinder-Entitäten übernommen werden.

Zusätzlich zu den bereits genannten Funktionen muss es möglich sein 
die Inhalte der Entitäten zu exportieren. Dies geschieht in Form einer C-Datei. 

Die C-Datei hat folgende Struktur: 
\lstinputlisting[language=C,caption={C-Datei Beispiel},captionpos=b]{listings/c-file.c}
Anzumerken ist, dass in der C-Datei als deaktiviert markierte Tarife oder Produkte nicht exportiert werden sollen. Zusätzlich sollen nur die Inhalte der aktuell ausgewählten Entität übernommen werden, also ohne Kind Daten.
Diese Funktion ist besonders wichtig, da die Inhalte der C-Datei von anderen Anwendungen benötigt werden. Besonders sticht hier SST-Referenz heraus.

Die letzten Funktionalitäten dienen allein den \texttt{Überschüssen}. Die \texttt{Überschüsse} werden wie bereits im vorherigem Kapitel erwähnt, jedes Jahr neu berechnet.
Dementsprechend benötigt es diverse Funktionen um dies zu erleichtern. 

Dazu gehört die Neuberechnung des Gesamtzins. Die Umsetzung ist so geplant, dass der neue Gesamtzins als Input eingegeben wird.
Nun werden alle \texttt{Überschüsse} mit einem neuen Eintrag angelegt, der bis Ende des folge Jahre gültig ist. Die gültigBis Property des vorherigen Eintrags wird folglich auf das Ende des aktuellen Jahres geändert.
Der neue Eintrag ist letztlich eine vollständige Kopie des alten Eintrags, jedoch mit dem neuen Gesamtzins. Dies ist der Fall, falls sich die Gesamtverzinsung in einem Jahr ändert.

Nun gibt es noch die Möglichkeit, dass sich der Gesamtzins nicht ändert. In diesem Fall müssen nicht alle Einträge geändert werden, sondern nur ausgewählte. 
Dafür muss es möglich sein einen Eintrag fortzuschreiben wie bei der Änderung des Gesamtzins. Also ein neuer Eintrag mit neuer Gültigkeit und aktualisieren des alten Eintrags.
Im neuen Eintrag werden dann die Werte die sich geändert haben, manuell angepasst. Nach der Anpassung müssen alle unveränderten Einträge ebenfalls fortgeschrieben werden. 
Also muss es eine Funktion geben die alle aktuellen Einträge kopiert und die Gültigkeit erneuert.

\section{Technologie Auswahl}
\subsection{Frontend: Angular}
Als Frontend-Framework wurde Angular ausgewählt. Diese Entscheidung war naheliegend, da Angular das am weitesten verbreitete Framework innerhalb der Nürnberger ist und mein Chef ausdrücklich den Einsatz dieses Frameworks gewünscht hat.

Auch aus meiner Sicht ist Angular eine sinnvolle Wahl. Es ist mein bevorzugtes Framework, da ich bereits mehrere eigene Projekte damit umgesetzt habe und auch innerhalb der Nürnberger umfangreiche Erfahrungen damit sammeln konnte. 
Dadurch bin ich mit der Entwicklungsumgebung, den Werkzeugen und Best Practices bestens vertraut.

Zwar hätte ich es interessant gefunden, mich in andere Frameworks wie Vue.js oder Svelte einzuarbeiten, doch die Entscheidung für Angular bringt klare Vorteile: 
Meine bestehende Erfahrung ermöglicht eine schnellere Entwicklung und einen saubereren Code. Darüber hinaus erleichtert die Verbreitung von Angular innerhalb des Unternehmens die spätere Einarbeitung und Zusammenarbeit mit Kollegen, falls das Projekt übergeben oder erweitert werden muss.
\subsection{Backend: Spring Boot vs ASP.Net}
Im weiteren Verlauf der Planung wurde darüber diskutiert, ob die Umsetzung des Backends lieber mit Java oder C\# erfolgen soll. 
Sowohl Java als auch C\# sind objektorientierte Programmiersprachen die ausreichend Tools zur Entwicklung von Web- und CRUD- (Create, Read, Update, Delete) Anwendungen bieten.
Das heißt das jegliche Sprache für die von uns definierten Projektziele mehr als ausreichend ist.

Auf der Seite von Java stehen Spring Boot für die Web Funktionalität und Spring Data JPA für die Abstrahierung der Datenbank zur Verfügung.
C\# auf der anderen Seite bietet ASP.Net Core für die Web Funktionalität und Entity Framework Core für die Abstrahierung der Datenbank.
Da ich bereits reichlich Erfahrung in der Entwicklung mit C\#, sowie ASP.Net und EFC habe, war dieses auch meine persönlich Präferenz für das Projekt.
Innerhalb der Nürnberger ist jedoch C\# nicht weitreichend genutzt und stattdessen wird Java bevorzugt. Dieser Meinung ist auch mein Betreuer, jedoch lies er mir die Option offen, mit 
ausreichend Argumenten, auch C\# zu verwenden. 
Dementsprechend recherchierte ich die Vor- und Nachteile beider Technologien und stellte diese gegenüber.

Es gibt zwei starke Argumente die für die Verwendung von der Microsoft Umgebung rund um C\# sprechen.
Hier ist das Hauptaugenmerk auf meine weitreichende Erfahrung in der Entwicklung mit C\# und ASP.Net zu legen.
Die nicht notwendige Einarbeitung in eine neue Sprache und Umgebung würde definitiv die Entwicklung beschleunigen.
Ebenso würde das Vorwissen vermutlich auch die Qualität des Codes erhöhen, da ich bereits Methoden kenne, wie man Architektur, Entität-Beziehungen und Datenbankzugriffe effizient umsetzt.
Ein weiterer Vorteil ist C\# leicht bessere Performance und deutlich besseres Speichermanagement. 

Die genauen Zahlen sind in \cite{Debian2025} dargestellt. 
Betrachtet man die Algorithmen \textit{fannkuch-redux} und \textit{n-body}, zeigt sich, dass C\# effizienter ist als Java.

Beim \textit{fannkuch-redux} benötigt C\# durchschnittlich 23,04 Sekunden und 29.184~KB Speicher, 
während Java mit 28,38 Sekunden etwa 19~\% langsamer ist und mit 43.582~KB etwa 50~\% mehr Speicher verbraucht.

Beim \textit{n-body} liegt C\# mit 5,42 Sekunden Laufzeit und 28.946~KB Speicher ebenfalls vorn. 
Java benötigt hier 7,36 Sekunden (26~\% langsamer) und 42.226~KB (48~\% mehr Speicher).

In beiden Algorithmen ist C\# sowohl schneller als auch speichereffizienter. 
Java zeigt konstant einen höheren Speicherverbrauch, was auf die JVM-Architektur zurückzuführen sein könnte.

Das Hauptargument auf der Seite von Java ist jedoch die weitreichende Nutzung innerhalb der Nürnberger Versicherung. Dies resultiert in zu einem besseren Support innerhalb des Unternehmens.
Ebenso ist das Maintaining dieser Anwendungen simpler, da die meisten Entwickler hier bereits Erfahrung in Java haben. Dementsprechend ist es für einen möglichen nachfolgenden Entwickler leichter
in das Projekt einzusteigen und gegebenenfalls zu übernehmen.
Das war auch letztlich der überzeugenden Punkt für die Verwendung von Java. 

Die Argumente für C\# und weg von Java waren letztlich nicht stark genug.
Zusätzlich bedeutet das für mich auch eine Herausforderung, in der ich eine neue weitere Sprache und Umgebung lernen kann.

\subsection{Datenbank: MSSQL vs Oracle}
Der nächste Schritt war die Auswahl der Datenbank. Zur Diskussion standen MSSQL und Oracle, beides relationale SQL-Datenbanken, 
die die notwendigen Funktionen für das Projekt bieten. Eine grundlegende Anforderung meines Chefs war, dass es sich um eine SQL-Datenbank handeln muss.

In einem gemeinsamen Gespräch mit den internen Architekten wurde schließlich entschieden, dass MSSQL die bessere Wahl für dieses Projekt ist.

Auch wenn Oracle eine gängige Datenbank innerhalb der Nürnberger ist, ist sie vergleichsweise teuer und zu überdimensioniert für die Anforderungen des Projekts.
MSSQL hingegen ist kostengünstiger und ist für die Datenmengen und möglichen Herausforderungen des Projektes mehr als ausreichend. \cite{petri2005comparison}

\subsection{Architektur: Hexagonal}
Der letzte Planungsabschnitt ist die Auswahl der Architektur. Hier wurde mir komplette Freiheit gelassen.
Die einzige Bedingung hierbei war, dass die Logik aufgeteilt wird und somit eine Schichtentrennung erfolgt.

Hierbei fiel meine Entscheidung auf die Hexagonale Architektur.

Das Grundprinzip der Hexagonalen Architektur ist es die Anwendung in verschiedene Schichten aufzuteilen.
Diese unterschiedlichen Schichten sind von außen nach innen abhängig von einander. Das heißt das jede innere Schicht nichts von einer Schicht weiter außen weiß.
In der Regel ist die Model-Schicht die innerste Schicht.
Dann kommt die Service-Schicht mit der Anwendungslogik und die Controller-Schicht mit den APIs als Schnittstelle zum User-Interface.

Im meinem konkreten Fall bedeutet das, dass die Web-Layer alle Schnittstellen anbietet die mein Fronted benötigt. Darunter zählen \texttt{GET}-Anfragen um Daten zu bekommen oder
\texttt{POST}-Anfragen um Daten zu speichern.
Die \texttt{Domain-Layer} ist die eigentliche Logik der Anwendung. Hier finden alle Berechnungen und Datenverarbeitungen, wie das Ändern der \texttt{Überschüsse} oder die Generierung der C-Datei, statt.
Die \texttt{Data-Layer} ist die Schnittstelle zur Datenbank. Hier werden alle Datenbankzugriffe durchgeführt. In dieser Schicht sind auch alle Entitäten definiert.

In \ref{fig:architecture} ist die Architektur des Systems dargestellt. Der Nutzer interagiert mit dem Frontend. Die Anfrage wird an die \texttt{Web-Layer} geschickt. 
Diese ruft eine Methode im Service auf. Der Service in der \texttt{Domain-Layer} führt die Anfrage aus und ruft, falls Daten benötigt oder bearbeitet werden, die \texttt{Data-Layer} auf.
Die \texttt{Data-Layer} greift auf die Datenbank zu und gibt die Daten zurück.

\begin{figure}[htbp]
  \resizebox{\textwidth}{!}{\begin{tikzpicture}[node distance=2cm, align=center]

  % User and Frontend
  \node (user) [block, text width=4em] at (-3cm, 0) {User};
  \node (frontend) [block, text width=6em, below=2cm of user] {Frontend Angular};
  
  % Backend
  \node (webLayer) [layer, below=2cm of frontend] {Web \\ Layer};
  \node (domainLayer) [layer, right=3.5cm of webLayer] {Domain \\ Layer};
  \node (dataLayer) [layer, right=3.5cm of domainLayer] {Data \\ Layer};
  
  % Database
  \node (database) [database, above=4cm of dataLayer] {Database \\ MSSQL};
  
  % Fit Backend Box
  \node[draw, rounded corners, inner sep=0.7cm, fit=(webLayer) (domainLayer) (dataLayer), label={[anchor=north]north:Backend (Java, Springboot)}] (backend) {};
  
  % Arrows with labels
  \draw [arrow] (user) -- node[left] {interacts} node[right] {with} (frontend);
  \draw [arrow] (frontend) -- node[left] {http} node[right] {request} (webLayer);
  \draw [arrow] (webLayer) -- node[above] {call service} node[below] {methods} (domainLayer);
  \draw [arrow] (domainLayer) -- node[above] {ask for data} (dataLayer);
  \draw [arrow] (dataLayer) -- node[left] {access} node[right] {data} (database);
    
\end{tikzpicture}}
  \caption{Architektur des Systems}
  \label{fig:architecture}
\end{figure}

Diese Architektur bietet mehrere Vorteile. Durch die genannte Aufteilung ist es einfacher, einzelne Schichten zu ändern und auszutauschen. 
So ist es beispielsweise dem \texttt{Web-Layer} egal, welches Frontend verwendet wird, und dem \texttt{Data-Layer}, welche Datenbank zum Einsatz kommt.
Ebenfalls kann die Business-Logik unabhängig von den anderen Schichten getestet werden.
Das wichtigste Argument für diese Architektur war jedoch, dass sie zum einen die Anforderung der geforderten Trennung erfüllt und die einzige Architektur ist, die ich bereits in der Praxis verwendet habe.