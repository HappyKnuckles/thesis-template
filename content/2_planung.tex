\chapter{Projektplanung}\label{ch:data}
\section{Projektziele}
Die groben Projektziele lassen sich aus der Projektaufgabe ableiten. 
Die Webanwendung soll das Hinzufügen von neuen Tarifen und Produkten ermöglichen. Dies soll so umgesetzt werden, das die zugehörigen Kind-Entitäten nur aus der Eltern-Entität hinzugefügt werden können.
Eine weiter Funktionalität ist das Anlegen von Tarifen anhand eines bestehenden Tarifs. 
Hierbei sollen die Daten des bestehenden Tarifs - auch Kinder - übernommen werden und der neue Tarif kann dann angepasst werden.
Eine ebenso wichtige Anforderung ist das mögliche deaktivieren von Tarifen und Produkten. Dies soll die Inaktivität der jeweiligen Entität darstellen, diese werden dann z.B. nicht mehr im Export berücksichtigt.
Sobald ein Tarif oder Produkt inaktiv ist, soll es nun möglich sein diese zu löschen. Hierbei ist es wichtig, dass die zugehörigen Kinder ebenfalls gelöscht werden. 
Die Überschüsse sollen als einzige Entität komplett bearbeitet werden können, da diese jährlich neu berechnet werden. Eine andere Möglichkeit um diese zu ändern soll im allgemeinen durch die Gesamtzinsänderung erfolgen.
Dies ist so geplant, dass man die Gesamtzinsänderung eingibt und die Überschüsse den neuen Zins übernehmen. Die letzte aber womöglich einer der wichtigsten Anforderungen ist das Exportieren, der jeweiligen Entitäten.
Die Daten sollen in einer C-Datei eingepflegt werden in Form eines Arrays. Zusätzlich zu den Properties soll auch die Anzahl der Elemente mit exportiert werden. Kinder sollen nicht berücksichtigt werden.
Diese Funktionalität ist besonders wichtig, da diverse anderen Anwendungen diese Dateien benötigen.

\section{Technologie Auswahl}
\subsection{Frontend: Angular}
\subsection{Backend: Spring Boot vs ASP.Net}
Im weiteren Verlauf der Planung wurde darüber diskutiert, ob die Umsetzung des Backends lieber mit Java oder C\# erfolgen soll. 
Sowohl Java als auch C\# sind objektorientierte Programmiersprachen die ausreichend Tools zur Entwicklung von Web- und CRUD- (Create, Read, Update, Delete) Anwendungen bieten.
Das heißt das jegliche Sprache für die von uns definierten Projektziele mehr als ausreichend ist.
Auf der Seite von Java stehen Spring Boot für die Web Funktionalität und Spring Data JPA für die Abstrahierung der Datenbank zur Verfügung.
C\# auf der anderen Seite bietet ASP.Net Core für die Web Funktionalität und Entity Framework Core für die Abstrahierung der Datenbank.
Da ich bereits reichlich Erfahrung in der Entwicklung mit C\#, sowie ASP.Net und EFC habe, war dieses auch meine persönlich Präferenz für das Projekt.
Innerhalb der Nürnberger ist jedoch C\# nicht weitreichend genutzt und stattdessen wird Java bevorzugt. Dieser Meinung ist auch mein Betreuer, jedoch lies er mir die Option offen, mit 
ausreichend Argumenten, auch C\# zu verwenden. 
Dementsprechend recherchierte ich die Vor- und Nachteile beider Technologien und stellte diese gegenüber.
Es gibt zwei starke Argumente die für die Verwendung von der Microsoft Umgebung rund um C\# sprechen.
Hier ist das Hauptaugenmerk auf meine weitreichende Erfahrung in der Entwicklung mit C\# und ASP.Net zu legen.
Die nicht notwendige Einarbeitung in eine neue Sprache und Umgebung würde definitiv die Entwicklung beschleunigen.
Ebenso würde das Vorwissen vermutlich auch die Qualität des Codes erhöhen, da ich bereits Methoden kenne, wie man Architektur, Entität-Beziehungen und Datenbankzugriffe effizient umsetzt.
Ein weiterer Vorteil ist C\# leicht bessere Performance und deutlich besseres Speichermanagement. https://benchmarksgame-team.pages.debian.net/benchmarksgame/fastest/csharp.html
Das Hauptargument auf der Seite von Java ist die weitreichende Nutzung innerhalb der Nürnberger Versicherung. Dies resultiert in zu einem besseren Support innerhalb des Unternehmens.
Ebenso ist das maintaining dieser Anwendungen simpler, da die meisten Entwickler hier bereits Erfahrung in Java haben. Dementsprechend ist es für einen möglichen nachfolgenden Entwickler leichter
in das Projekt einzusteigen und gegebenenfalls zu übernehmen.
Das war auch letztlich der überzeugenden Punkt für die Verwendung von Java. Die Argumente für C\# weg von Java waren letztlich nicht stark genug.
Zusätzlich bedeutet das für mich auch eine neue Herausforderung, in der ich eine neue zusätzliche Sprache und Umgebung lernen kann.

\subsection{Datenbank: MSSQL vs Oracle}
Der nächste Schritt war die Auswahl der Datenbank. Hier standen MSSQL und Oracle zur Auswahl.
\subsection{Architektur: Hexagonal}

