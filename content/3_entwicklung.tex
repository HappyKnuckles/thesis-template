\chapter{Entwicklung}\label{ch:method}
\section{Backend}
\subsection{Projektstruktur}

\subsection{Entitäten-Implementierung}
Die Entitäten sind das Grundgestein des Backends und auch der Datenbank.
Hier habe ich zuerst mit meinem Chef in mehreren Gesprächen definiert, welche Entitäten wir für das Projekt benötigen.

Die zu implementierenden Entitäten belaufen sich auf: Tarif, Tarifbaustein, Tarifbaustein-Pricing, Produkt, Produkt-Pricing, Tafelsystem und Überschuss.

Im folgenden wurden die benötigten Felder für die Entitäten definiert. 
// Hier die Klassen einfügen

Die Entitäten wurden als Klassen in dem Projekt 'Domain' implementiert. Hierfür wurden zuerst die Properties der Klasse hinzugefügt und dann Getter und Setter geschrieben.
Anschließend wurden die Entitäten mit der JPA-Annotation '@Entity' versehen, um die Verbindung zur Datenbank zu ermöglichen.

Der nächste Schritt war es die Beziehungen zwischen den Entitäten hinzuzufügen. Hierbei gab es mehrere einfache 1:n Beziehungen: Tarif zu Tarifbaustein, Tarifbaustein zu Tarifbaustein-Pricing und Produkt zu Produkt-Pricing.

Die Beziehung zwischen Tarifbaustein und Tafelsystem sowie Überschuss waren jedoch ein bisschen komplexer. 
Hier hatte ich ursprünglich, anhand der Vorbesprechungen, jeweils drei n:m Beziehung geplant.
Dementsprechend habe ich diese Beziehungen als je drei Listen implementiert, die die Keys der jeweils anderen Entitäten als Fremdschlüssel nutzen.
Dies war für mich eher schwierig umzusetzen, da ich in Java noch nicht mit den Annotationen '@ManyToMany' und '@JoinTable' gearbeitet hatte.

In weiteren Gesprächen mit meinen Chef wurde dann letztlich deutlich, das es keine drei n:m Beziehungen sind, sondern drei 1:n Beziehungen.
Obwohl die Beziehungen bereits implementiert waren, fand ich die Notwendigkeit die Entitäten anzupassen, gut. Diese Änderung nimmt nämlich einiges an Komplexität aus dem Projekt heraus und
erleichterte mir somit alle weiteren Schritte zur Implementierung der Services und Controller.

Als die Beziehungen fertig definiert wurden, habe ich nun die Annotationen // TODO hinzufügen der richtigen Annotationen hinzugefügt, damit bei 
der JSON-Serialisierung die Beziehungen korrekt dargestellt werden und keine Circular-Dependencies entstehen.

Im späteren Verlauf wurde noch eine Funktion zum String-Serialisieren der Properties hinzugefügt. Dies wird benötigt um die Entitäten in die C-Datei schreiben zu können.

Zusätzlich wurden alle Getter und Setter entfernt, da ich im Seminar von dem Package 'Lombok' erfahren habe und die '@Getter' und '@Setter' Annotationen die Klassen leichter lesbar machen.
\subsection{Interface-Initialisierung}
\subsection{Service-Implementierung}
\subsection{Controller-Implementierung}
\section{Frontend}
\section{Datenbank}
\
