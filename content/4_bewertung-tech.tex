\chapter{Bewertung der Technologien}
\section{Angular}
Angular ist sowohl privat als auch beruflich mein bevorzugtes Frontend-Framework.
Die vorgegebene Struktur die Angular vorgibt, erinnert stark an Programmierprinzipien, die man normalerweise in Backend-Sprachen findet.
Dadurch ist es mir möglich, sehr vertraut zu arbeiten und das Single-Responsibility-Prinzip strikt umzusetzen.
Dementsprechend fand ich die Arbeit mit Angular sehr angenehm und produktiv.

Als Kritikpunkt möchte ich jedoch anmerken, dass ich bedingt durch meine Vorerfahrung vergleichsweise wenig neues lernen konnte.
Die neuen Erfahrungen begrenzen sich hier lediglich auf die Integration von Swagger-Definitionen in das Frontend und eine vermehrte Verwendung von
Directives um bei bestehenden Komponenten zusätzliche Funktionen hinzuzufügen.

Alles in einem ist Angular jedoch ein sehr mächtiges Framework, das ich jedem Entwickler empfehlen kann.
Vor allem denen, die eher aus der Backend-Entwicklung kommen und eine klare Trennung bevorzugen.

\section{Spring Boot/Java}
Spring Boot und Java waren zu Beginn des Projekts Technologien, auf die ich mich nicht besonders gefreut habe – vor allem, 
weil ich wenig Erfahrung mit Java und dessen Ökosystem hatte und die Sprache stets als eher umständlich und altmodisch empfand. 
Dennoch erachtete ich die Entscheidung, Java und Spring Boot als Backend-Technologie einzusetzen, als durchaus sinnvoll, da ich dadurch vieles Neues lernen und zahlreiche Parallelen zur C\#-Welt erkennen konnte.

Letztlich bieten Java und Spring Boot nahezu dieselben Möglichkeiten wie C\# und Entity Framework Core (EFC), 
was mir einen schnellen Einstieg und eine zügige Einarbeitung ermöglichte. 

Trotz der offensichtlichen Ähnlichkeiten bewerte ich Java insgesamt etwas weniger positiv. Besonders als Neuling empfinde ich die Syntax als anspruchsvoller, 
und der Umgang mit Libraries und Frameworks gestaltet sich oft umständlicher – etwa aufgrund des Unterschieds zwischen 
primitiven Datentypen und den entsprechenden Klassen. Hier hat man zum Beispiel die Wahl, ob man double oder Double verwendet.
Das ist meiner Ansicht nach unnötig kompliziert und führt zu unnötigen Fehlern.

Meine anfänglichen Vorbehalte haben sich im Verlauf des Praktikums eher relativiert, sodass ich insgesamt positive Erfahrungen mit Java und Spring Boot sammeln konnte. 
Das Praktikum ermöglichte mir, mich der Technologie anzunähern – auch wenn ich bei einem ähnlichen Use-Case in Zukunft wahrscheinlich eher zu C\# und EFC tendieren würde.
\section{MSSQL}
Da ich die Datenbank bisher noch nicht implementieren konnte, kann ich hierzu keine abschließende Bewertung abgeben.
Ich möchte jedoch anmerken, dass ich mit MSSQL ohnehin kaum direkt hätte arbeiten müssen, 
da die Einrichtung von einer separaten Abteilung übernommen wird und die Konfiguration über Spring Boot erfolgt, anstatt direkt in der Datenbank.