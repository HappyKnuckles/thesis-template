\chapter{Bewertung und Bezug}
Im Rahmen des Studiums erhält man eine fundierte Einführung in verschiedene Technologien und deren Grundlagen. 
Dazu gehören Programmierkenntnisse, grundlegendes Wissen über Datenbanken, eine Einführung in 
Algorithmen und Datenstrukturen sowie die Basisprinzipien der Informatik.

Diese Inhalte bilden zwar ein solides Grundwissen, sind jedoch oft weit von den Anforderungen und der Arbeitsweise in der Praxis entfernt.
Diese Erfahrung konnte ich unter anderem während meiner Tätigkeit bei der Nürnberger Versicherung machen.

Die an der Hochschule vermittelte Basis erleichtert zweifellos die initiale Einarbeitung in neue Technologien und Projekte. 
Allerdings sind viele der im Studium behandelten Themen in der Praxis stark abstrahiert oder werden in einer anderen Form umgesetzt. 
Dies erfordert eine ständige Weiterentwicklung und Lernbereitschaft, um mit aktuellen Technologien klarzukommen.

Ein gutes Beispiel dafür ist der Umgang mit Datenbanken. Im Studium lernt man den Umgang mit SQL und versteht, wie relationale Datenbanken 
strukturiert sind. In der Praxis wird SQL jedoch häufig durch Frameworks abstrahiert. 
Technologien wie Entity Framework oder Spring Boot generieren SQL automatisch und ermöglichen den Zugriff auf Datenbanken durch 
Abstraktionsebenen wie ORMs (Object-Relational Mappers). Dadurch tritt das direkte Arbeiten mit SQL oft in den Hintergrund, und der Fokus liegt 
vielmehr auf der korrekten Integration und Konfiguration dieser Frameworks.

Ähnlich verhält es sich mit Algorithmen und Datenstrukturen. Während diese Konzepte im Studium intensiv behandelt werden, 
um ein tiefes Verständnis für die Grundlagen der Informatik zu vermitteln, begegnen sie einem in der Praxis meist nur indirekt. 
Häufig bieten Frameworks und Bibliotheken bereits optimierte Implementierungen für viele dieser Konzepte. Der Fokus liegt dann darauf, die eigenen Methoden der Programmiersprachen für
die Implementierung dieser Konzepte zu kennen und richtig zu nutzen.

Trotz dieser Abstraktionen ist das im Studium vermittelte Wissen sehr wichtig, da es hilft, 
die Hintergründe und Funktionsweisen moderner Technologien zu verstehen. Dieses Verständnis erleichtert es, 
sich schnell in neue Frameworks oder Technologien einzuarbeiten und deren Grenzen sowie Einsatzmöglichkeiten besser zu erkennen.

Zusammenfassend lässt sich sagen, dass das Studium eine wichtige Grundlage legt, auf der man in der Praxis aufbauen kann. 
Die wahre Herausforderung besteht jedoch darin, diese Basis kontinuierlich zu erweitern, neue Technologien zu erlernen 
und sich den Anforderungen der beruflichen Realität anzupassen.

