\chapter{Fazit}

Im Rahmen meines Praktikums bei der Nürnberger Versicherung konnte ich wertvolle Erfahrungen in der Softwareentwicklung sammeln. 
Die Aufgabe, eine Webanwendung zur Verwaltung von Tarifen und Produkten zu entwickeln, stellte eine spannende Herausforderung dar, da ich mit Datenmengen und Entitäten zu tun hatte, die einem im Alltag nicht wirklich begegnen. d
Zusätzlich ermöglichte diese, meine Kenntnisse in verschiedenen Technologien zu vertiefen oder neue Fähigkeiten zu erlernen.

Die Planung und Implementierung der Anwendung umfasste sowohl das Frontend mit Angular als auch das Backend mit Spring Boot und Java. 
Dabei konnte ich die Vorteile der Hexagonalen Architektur nutzen, um eine klare Trennung der Logikschichten zu gewährleisten. 
Die Entscheidung für MSSQL als Datenbank kann ich leider noch nicht beurteilen.

Besonders hervorzuheben ist die Dynamik bei der Entwicklung. Hier konnte ich in die Rolle des Tech-Leads schlüpfen und das Projekt anhand den Anforderungen frei planen und definieren.
Dies lernte mir besonders, das selbstständige Arbeiten und die Verantwortung für ein Projekt zu übernehmen.

Die Arbeit an diesem Projekt hat mir gezeigt, wie wichtig eine sorgfältige Planung und die genaue Definition von Projektzielen sind.
Diese Planung ist maßgeblich für den Erfolg eines Projektes und vor allem für die Vermeidung von unnötigen Fehlern und Mehrarbeit. 
Die enge Zusammenarbeit mit meinem Chef und die regelmäßigen Abstimmungen haben dazu beigetragen, dass das Projekt größtenteils erfolgreich umgesetzt werden konnte.

Während des Praktikums habe ich gelernt, wie wichtig es ist, sich schnell in neue Technologien und Frameworks einzuarbeiten. 
Ich habe meine Fähigkeiten in Angular und Spring Boot erheblich verbessert und ein tieferes Verständnis für die Arbeit bei der Nürnberger entwickelt. 
Darüber hinaus habe ich gelernt, wie man spezifische Geschäftslogik in einer Webanwendung implementiert und dabei auf Skalierbarkeit und Wartbarkeit achtet.

Ein weiterer wichtiger Lernaspekt war das Demonstrieren des aktuellen Standes. 
Ich habe erfahren, wie wichtig regelmäßige Meetings und klare Absprachen sind, um Missverständnisse zu vermeiden und den Projektfortschritt sicherzustellen.

Allerdings gab es auch einige Herausforderungen und negative Aspekte. Darunter zählen zum einen typische Probleme, die in großeren Unternehmen auftauchen, wie zum Beispiel das Beantragen von Berechtigungen oder auch VPN- oder Proxy-Probleme.
Zum anderen gab es auch bei der Technik Herausforderungen, wie zum Beispiel die automatische Integration von Swagger oder generell anfangs die Arbeit mit Java und Spring Boot.

Zusammenfassend war das Praktikum eine bereichernde Erfahrung, die mir nicht nur technische Fähigkeiten vermittelt hat, sondern auch Einblicke in die Arbeitsweise und die Anforderungen eines großen Unternehmens wie der Nürnberger Versicherung gegeben hat. 
Trotz einiger Herausforderungen und frustrierender Momente bin ich zuversichtlich, dass die Implementierung meines Projektes seinen Nutzen innerhalb der Nürnberger findet und die Prozesse in einigen Bereichen vereinfacht werden können. 
Das Projekt wird auch nach dem Praktikum weitergeführt, verbessert und erweitert.